\documentclass{ieeeojies}
\usepackage{cite}
\usepackage{amsmath,amssymb,amsfonts}
\usepackage{algorithmic}
\usepackage{graphicx}
\usepackage{textcomp}



\def\BibTeX{{\rm B\kern-.05em{\sc i\kern-.025em b}\kern-.08em
    T\kern-.1667em\lower.7ex\hbox{E}\kern-.125emX}}

\begin{document}
\tableofcontents

\title{Informe de Análisis y Diseño Parcial 1}
\author{\uppercase{YESIKA-MILENA CARVAJAL-DÍAZ}\authorrefmark{1}, \uppercase{ · NICOLL-CAROLINE CHAZATAR-YAMPUEZAN}\authorrefmark{2}, \uppercase{  · ANDRES-FELIPE ZULUAGA-ORTÍZ}\authorrefmark{3}}


\address[1]{Universidad de Antioquia, Medellín, Colombia.}
\address[2]{Ingeniería Electrónica, Facultad de Ingeniería, Universidad de Antioquia.}




\begin{abstract}
\end{abstract}


\begin{keywords}

\end{keywords}



\titlepgskip=-15pt

\maketitle

\section{Introduction}

\section{RESÚMEN}

\section{OBJETIVOS}

\subsection{GENERALES}

\subsection{ESPECÍFICOS}

\section{DESARROLLO}

\begin{enumerate}
\item \textbf{[20] Investigar y explicar con sus propias palabras como utilizar el circuito integrado 74HC595. En esta parte mostrar un ejemplo de uso del circuito integrado primero de forma independiente con pulsadores o switches y luego usando Arduino, y detallar en el informe la utilidad que puede darle a este elemento para la solución del problema. Nota: puede ser de gran ayuda dar una revisión a la hoja de datos.} 


Los pines de alimentación del integrado son el pin 8 para tierra y 16 para Vcc.
Se debe generar una diferencia de potencial en los pines MA y OE para habilitar las salida del integrado, por lo que el pin OE se conecta a tierra y MA a Vcc.

Los pines SH-CP, DS, ST-CP controlan el ingreso de los bits que se reproducen en las salidas del integrado. El pin DS da el valor del bit, el SH-CP es la señal para tomar el bit de DS, y STCP muestra los bits almacenados previamente en las salidas del integrado. Las señales reproducidas inician desde la salida Q0 hasta Q7.

El circuito integrado es útil en la realización de la práctica en la creación del bloque Sistema que paraleliza los datos. El 74HC595 nos ayuda a paralelizar los bytes en sus 8 puertos de salida para que estos entren al Sistema de desencriptación.




\item \textbf{[20] Realizar una comunicación entre dos Arduinos usando puertos digitales, esto con el fin de que pueda probar como enviar bits desde uno y ser recibidos por el otro de forma
exitosa. Para esto, incluya en el informe de forma ordenada que consideraciones tuvo en
cuenta para esta parte, y en una carpeta en el repositorio incluya los códigos usados para
cada Arduino.}




\end{enumerate}

\section{CONCLUSIONES}

\appendices

\section*{BIBLIOGRAFÍA}



%% these lines used to import a separate ".bib" for the bibliografy.
\bibliographystyle{bibliography/IEEEtranIES}
\bibliography{bibliography/IEEEabrv,bibliography/mybibfile}

\EOD

\end{document}
